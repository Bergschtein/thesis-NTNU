\documentclass[../../thesis.tex]{subfiles}

\begin{document}

Time series generation (TSG) focuses on learning the distribution of time series data through generative modeling. State-of-the-art approaches in TSG, such as TimeVQVAE, utilize vector quantization-based tokenization to effectively model complex distributions. These models learn discrete latent representations of time series, and our research aims to enhance the \textit{expressiveness} of these representations. We introduce a novel framework for integrating non-contrastive self-supervised learning into the tokenization model, termed NC-VQVAE. Our model is evaluated on a subset of the UCR archive, using SVM and KNN accuracies to assess the expressiveness of the latent representations, and IS, FID, and CAS metrics to evaluate the quality of the generated samples. Our results demonstrate that NC-VQVAE learns representations that improve classification and clustering of time series, while also enhancing the quality of synthetic samples.
\end{document}



