\documentclass[../../thesis.tex]{subfiles}

\begin{document}

Tidsrekkegenerering (TSG) fokuserer på å lære fordelingen til tidsrekker ved generativ modellering. De beste metodene i TSG, som TimeVQVAE, utnytter vektorkvantisering for å effektivt modellere komplekse fordelinger. Disse modellene lærer diskrete latente representasjoner av tidsrekker, og vår forskning forsøker å gjøre disse mer \textit{uttrykksfulle}. Vi introduserer et nytt rammeverk for integrering av ikke-kontrastiv selvstyrt læring i VQVAE. Vår modell, NC-VQVAE, evalueres på en undermengde av UCR-arkivet. Vi bruker SVM- og KNN-nøyaktighet til å bedømme uttrykksfullheten til de latente representasjonene, samt IS-, FID- og CAS-beregninger til å bedømme kvaliteten til de genererte tidsrekkene. Resultatene fører oss til konklusjonen at NC-VQVAE lærer representasjoner som forbedrer klassifisering og klynging av tidsrekker, samtidig som genereringskvaliteten øker.
\end{document}